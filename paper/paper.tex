
\documentclass[a4paper,USenglish]{lipics-v2021}
%This is a template for producing LIPIcs articles. 
%See lipics-v2021-authors-guidelines.pdf for further information.
%for A4 paper format use option "a4paper", for US-letter use option "letterpaper"
%for british hyphenation rules use option "UKenglish", for american hyphenation rules use option "USenglish"
%for section-numbered lemmas etc., use "numberwithinsect"
%for enabling cleveref support, use "cleveref"
%for enabling autoref support, use "autoref"
%for anonymousing the authors (e.g. for double-blind review), add "anonymous"
%for enabling thm-restate support, use "thm-restate"
%for enabling a two-column layout for the author/affilation part (only applicable for > 6 authors), use "authorcolumns"
%for producing a PDF according the PDF/A standard, add "pdfa"

%\pdfoutput=1 %uncomment to ensure pdflatex processing (mandatatory e.g. to submit to arXiv)
%\hideLIPIcs  %uncomment to remove references to LIPIcs series (logo, DOI, ...), e.g. when preparing a pre-final version to be uploaded to arXiv or another public repository

%\graphicspath{{./graphics/}}%helpful if your graphic files are in another directory

\bibliographystyle{plainurl}% the mandatory bibstyle

\title{Subsidiary Recursion in Coq} %TODO Please add

%\titlerunning{Dummy short title} %TODO optional, please use if title is longer than one line

\author{Aaron Stump}{Computer Science Dept., The University of Iowa, USA \and \url{http://www.cs.uiowa.edu/~astump/}}{aaron-stump@uiowa.edu}{http://orcid.org/0000-0002-9720-0003}{}%TODO mandatory, please use full name; only 1 author per \author macro; first two parameters are mandatory, other parameters can be empty. Please provide at least the name of the affiliation and the country. The full address is optional. Use additional curly braces to indicate the correct name splitting when the last name consists of multiple name parts.

\author{Alex Hubers}{Computer Science, The University of Iowa, USA}{alexander-hubers@uiowa.edu}{}{}

\author{Christopher Jenkins}{Computer Science, The University of Iowa, USA}{alexander-hubers@uiowa.edu}{http://orcid.org/
0000-0002-5434-5018}{}

\author{Benjamin Delaware}{Computer Science, Purdue University, USA \and \url{https://www.cs.purdue.edu/homes/bendy/}}{bendy@purdue.edu}{}{}

\authorrunning{A. Stump, A. Hubers, C. Jenkins, and B. Delaware} %TODO mandatory. First: Use abbreviated first/middle names. Second (only in severe cases): Use first author plus 'et al.'

\Copyright{Aaron Stump, Alex Hubers, Christopher Jenkins, and Benjamin Delaware} %TODO mandatory, please use full first names. LIPIcs license is "CC-BY";  http://creativecommons.org/licenses/by/3.0/

\ccsdesc[100]{Software and its engineering~Recursion}
\ccsdesc[100]{Software and its engineering~Polymorphism}

\keywords{strong functional programming, recursion schemes, positive-recursive types, impredicativity} %TODO mandatory; please add comma-separated list of keywords

\category{} %optional, e.g. invited paper

\relatedversion{} %optional, e.g. full version hosted on arXiv, HAL, or other respository/website
%\relatedversiondetails[linktext={opt. text shown instead of the URL}, cite=DBLP:books/mk/GrayR93]{Classification (e.g. Full Version, Extended Version, Previous Version}{URL to related version} %linktext and cite are optional

%\supplement{}%optional, e.g. related research data, source code, ... hosted on a repository like zenodo, figshare, GitHub, ...
%\supplementdetails[linktext={opt. text shown instead of the URL}, cite=DBLP:books/mk/GrayR93, subcategory={Description, Subcategory}, swhid={Software Heritage Identifier}]{General Classification (e.g. Software, Dataset, Model, ...)}{URL to related version} %linktext, cite, and subcategory are optional

%\funding{(Optional) general funding statement \dots}%optional, to capture a funding statement, which applies to all authors. Please enter author specific funding statements as fifth argument of the \author macro.

%\acknowledgements{I want to thank \dots}%optional

%\nolinenumbers %uncomment to disable line numbering



%Editor-only macros:: begin (do not touch as author)%%%%%%%%%%%%%%%%%%%%%%%%%%%%%%%%%%
\EventEditors{June Andronick and Leonardo da Moura}
\EventNoEds{2}
\EventLongTitle{Interactive Theorem Proving 2022}
\EventShortTitle{ITP 2022}
\EventAcronym{ITP}
\EventYear{2022}
\EventDate{2022}
\EventLocation{}
\EventLogo{}
\SeriesVolume{}
\ArticleNo{}
%%%%%%%%%%%%%%%%%%%%%%%%%%%%%%%%%%%%%%%%%%%%%%%%%%%%%%

\newcommand{\all}[2]{\forall\, #1.\, #2}


\begin{document}

\maketitle

%TODO mandatory: add short abstract of the document
\begin{abstract}
  This paper describes a functor-generic derivation in Coq of
  subsidiary recursion.  On this recursion scheme, inner recursions
  may be initiated within outer ones, in such a way that outer
  recursive calls may be made on results from inner ones.  The
  derivation utilizes a novel (necessarily weakened) form of
  positive-recursive types in Coq, dubbed retractive-positive
  recursive types.  A corresponding form of induction is also
  supported.  The method is demonstrated through several examples.
\end{abstract}

\section{Introduction: subsidiary recursion}
\label{sec:intro}

Central to interactive theorem provers like Coq, Agda, Isabelle/HOL,
Lean and others are terminating recursive functions over user-declared
inductive datatypes~\cite{agda,coq,isabelle-hol,lean}.  Termination is
usually enforced by a syntactic check for structural decrease.  This
structural termination is sufficient for many basic functions.  For
example, the well-known \texttt{span} function from Haskell's standard
library (\verb|Data.List|) takes a list and returns a pair of the
maximal prefix satisfying a given predicate \verb|p|, and the
remaining suffix:
\begin{verbatim}
span :: (a -> Bool) -> [a] -> ([a],[a])
span _ []    =  ([], [])
span p (x:xs) = if p x
                then let (ys,zs) = span p xs in (x:ys,zs)
                else ([],x:xs)
\end{verbatim}
\noindent The sole recursive call is \verb|span p xs|, and it occurs
in a clause where the input list is of the form \verb|x:xs|.  So the
input to the recursive call is a subdatum of the input, and hence this
definition is structurally decreasing.  In the appropriate syntax, it
can be accepted without additional effort by all the mentioned
provers.

This paper is about a more expressive form of terminating recursion,
called \textbf{subsidiary recursion}.  While performing an outer
recursion on some input \verb|x|, one may initiate an inner recursion
on \verb|x| (or possibly some of its subdata), preserving the
possibility of further invocations of the outer recursive function.
Let us see a simple example.  The function \verb|wordsBy| (from
\verb|Data.List.Extra|) breaks a list into its maximal sublists whose
elements do not satisfy a predicate \verb|p|.  For example,
\verb|wordsBy isSpace " good day "| returns \verb|["good","day"]|; so
\verb|wordsBy isSpace| has the same behavior as \verb|words| (from
\verb|Data.List|).  Code is in Figure~\ref{fig:wordsBy}.  The first
recursive call, \verb|wordsBy p tl|, is structural.  But in the
second, we invoke \verb|wordsBy p| on a value obtained from another
recursion, namely \verb|span|.  This is not allowed under structural
termination, but will be permitted by subsidiary recursion as derived
below.

\begin{figure}
\begin{verbatim}
wordsBy :: (a -> Bool) -> [a] -> [[a]]
wordsBy p [] = []
wordsBy p (hd:tl) =
  if p hd
  then wordsBy p tl 
  else let (w,z) = span (not . p) tl in
        (hd:w) : wordsBy p z
\end{verbatim}
\caption{Haskell code for \texttt{wordsBy}, demonstrating subsidiary recursion}
\label{fig:wordsBy}
\end{figure}

\subsection{Summary of results}

This paper presents a functor-generic derivation of terminating
subsidiary recursion and induction in Coq.  We should emphasize that
this is a derivation of this recursion scheme within the type theory
of Coq.  No axioms or other modifications to Coq of any kind are
required. Based on this derivation, we present several example
functions like \verb|wordsBy|, and prove theorems about them.  For
example, we prove the expected property that the sublists returned by
\verb|wordsBy| consist of elements satisfying \verb|not . p|.
For another, we give a definition of run-length encoding as a
subsidiary recursion using \verb|span|, and prove that encoding and
then decoding returns the original list.  Our approach applies to the
standard datatypes in the Coq library, and does not require switching
libraries or datatype definitions.

An important technical novelty of our approach is a derivation of a
weakened form of positive-recursive type in Coq.  Coq (Agda, and Lean)
restrict datatypes $D$ to be strictly positive: in the type for any
constructor of $D$, $D$ cannot occur to the left of any arrows.  Our
derivation needs to use positive-recursive types, where $D$ may occur
to the left of an even number (only) of arrows.  Coq requires strict
positivity because in the presence of other features of Coq's theory,
full positive-recursive types lead to a paradox~\cite{coquand88}.  We
present a way to derive a weakened form of positive-recursive type
that is sufficient for our examples (Section~\ref{sec:mu}).  The
weakening is to require only that $F\ \mu$ is a retract of $\mu$,
where $\mu$ is the recursive type and $F\ \mu$ its one-step unfolding.
Usually these types are isomorphic.  Hence, we dub these
\textbf{retractive-positive} recursive types.  This weakening has the
negative consequence of leading to a form of noncanonicity, but we
will see how to work around this.  Our definition of
retractive-positive recursive types makes essential use of impredicate
quantification, and hence cannot be soundly recapitulated in a
predicative theory like Agda's.

We begin by summarizing the interface our derivation provides for
subsidiary recursion (Section~\ref{sec:interface}), and then see
examples (Section~\ref{sec:examples}).  We next explain how the
interface is actually implemented (Section~\ref{sec:deriv}), including
our retractive-positive recursive types (Section~\ref{sec:mu}).  The
interface for subsidiary induction is covered next
(Section~\ref{sec:interfacei}), and example proofs using it
(Section~\ref{sec:examplesi}).  Related work is discussed in
Section~\ref{sec:related}.

All presented derivations have been checked with Coq version 8.13.2,
using command-line option \verb|-impredicative-set|.  The code may be
found as release \verb|itp-2022| (dated prior to the ITP 2022
deadline) at \url{https://github.com/astump/coq-subsidiary}.  The
paper references files in this codebase, as an aid to the reader
wishing to peruse the code.

\section{Interface for subsidiary recursion}
\label{sec:interface}

This section presents the interface our Coq development provides
for subsidiary recursion.

\subsection{The recursion universe}

Our approach is within a long line of work using ideas from universal
algebra and category theory to describe inductive datatypes and their
recursion principles.  On this approach, one describes transformations
to be performed on data as \emph{algebras}, which can then be
\textit{folded} over data.  The simplest form of algebras, namely
$F$-algebras, are morphisms from $F\ A$ to $A$, for carrier object
$A$.  From a programming perspective, an $F$-algebra is given input of
type $F\ A$, and must compute a result of type $A$.

Algebras for our subsidiary recursion are more complex.  First, for
reasons we will explain further below, the carrier of the algebra will
be a functor \verb|X : Set -> Set|.  Second, algebras have a specified
\emph{anchor type} \verb|C|, which we can think of as the datatype
\emph{as viewed by a containing recursion} or else, if this is a
top-level recursion, our development's version of the actual datatype
(e.g., \verb|List|).  The algebra is presented with:

\begin{itemize}
\item a type \verb|R : Set|, which will be this recursion's view of the datatype.
\item a function \verb|reveal : R -> C|, which reveals values of type \verb|R| as really having the anchor type.  
\item a function \verb|fold : FoldT Alg R|, which allows one to initiate subsidiary recursions in which the anchor type is \verb|R|.  Note that the algebra's anchor type is \verb|C|, but for subsidiary recursions the anchor type changes (to \verb|R|). We will present the type \verb|FoldT Alg R| below.
\item a function \verb|eval : R -> X R|, to use for making recursive calls, on any value of type \verb|R|.
\item and a \emph{subdata structure} \verb|d : F R|, where \verb|F| is the signature functor for the datatype.
\end{itemize}

\noindent The algebra is then required to produce a value of type \verb|X R|.

We will use Coq inductive types for the signature functors \verb|F| of
various datatypes, thus enabling recursions to use Coq's
pattern-matching on the subdata structure \verb|d|.  So the style of
coding against this interface retains a similar feel to structural
recursions.  Unlike with structural termination, though, the interface
here is type-based and hence compositional.  As we will see, it
supports nested and higher-order recursions.

As in previous work, we dub this interface a \emph{recursion
universe}~\cite{stump20}.  As in other domains using the term
``universe'', we have an entity (here, \verb|R|) from which one cannot
escape by using the available operations (for other cases:
the ordinal $\epsilon_0$ and $\omega^-$, the physical universe and traveling at
the speed of light).  Staying in the recursion universe is good,
because we may recurse (via \verb|eval|) on any value of type
\verb|R|.

Some points must still be explained, particularly why \verb|X| has type \verb|Set -> Set|,
and the definition of \verb|FoldT|.  Let us see these and other details next.

\subsection{The interface in more detail}

Let us consider two central files from our development.

\begin{figure}
  \begin{verbatim}
Inductive ListF(X : Set) : Set :=
| Nil : ListF X
| Cons : A -> X -> ListF X.

Definition inList : ListF List -> List := inn ListF.
Definition mkNil : List := inList Nil.
Definition mkCons (hd : A) (tl : List) : List := inList (Cons hd tl).
Definition toList : list A -> List.
Definition fromList : List -> list A.
\end{verbatim}
  \caption{Some basics from \texttt{List.v}, specializing the functor-generic derivation of subsidiary recursion to lists (\texttt{List.v})}
  \label{fig:listf}
\end{figure}

\subsubsection{Subrec.v}

  This file is parametrized by a signature functor \verb|F| of type
  \verb|Set -> Set|.  It provides the implementation of subsidiary
  recursion.  Two crucial values are \verb|Subrec : Set|, which is the
  type to use for subsidiary recursion; and
  \verb|inn : F Subrec -> Subrec|, which is to be used as a
  constructor for that type.  An important point, however, is that
  \verb|Subrec.v| does not provide an induction principle based on
  \verb|inn|.  Induction is derived later
  (Section~\ref{sec:interfacei}). \verb|Subrec.v| makes critical use
  of retractive-positive recursive types, to take a fixed-point of a
  construction based on \verb|F|.  We present these recursive
  types in Section~\ref{sec:mu} below.

  \subsubsection{List.v}

  This file specializes the development in \verb|Subrec.v| to the case
  of lists (parametrized by the type \verb|A| of elements).  In
  general, to use our development to get subsidiary recursion over
  some datatype, one will have a similar ``shim'' file.  The file
  defines the signature functor \verb|ListF|, shown in
  Figure~\ref{fig:listf}.  Using \verb|Subrec|, we then get a type
  \verb|List|.  This is not to be confused with the type \verb|list|
  of lists in Coq's standard library.  As noted previously, our
  development is meant to be used in extension of existing inductive
  datatypes, not replacing them.  The figure also shows constructors
  \verb|mkNil| and \verb|mkCons| for \verb|List|, and types for
  conversion functions between \verb|List| and \verb|list| (see
  Section~\ref{sec:deriv} for the code).
  

\subsection{Algebras for subsidiary recursion}

\verb|Subrec.v| also defines the notion of algebra that is used for
writing recursions.  The central definitions are in
Figure~\ref{fig:algf}.  \verb|KAlg| is the kind for the
type-constructor for algebras, as we see in the definition of
\verb|Alg|.  This type-constructor \verb|Alg| is a fixed-point of the
type \verb|AlgF|.  The fixed-point is taken using \verb|MuAlg|
(Section~\ref{sec:mu}), which implements our retractive-positive
recursive types at kind \verb|KAlg|.  Using \verb|Alg| will require
that \verb|AlgF| only uses its parameter \verb|Alg| positively.  We
will confirm this shortly.

\begin{figure}
\begin{verbatim}
Definition KAlg  : Type := Set -> (Set -> Set) -> Set.

Definition FoldT(alg : KAlg)(C : Set) : Set :=
  forall (X : Set -> Set) (FunX : Functor X), alg C X -> C -> X C.

Definition AlgF(Alg: KAlg)(C : Set)(X : Set -> Set) : Set :=
  forall (R : Set)
         (reveal : R -> C)        
         (fold : FoldT Alg R)
         (eval : R -> X R)      
         (d : F R),             
         X R.

Definition Alg : KAlg := MuAlg AlgF.

Definition fold : FoldT Alg Subrec.
Definition rollAlg :
  forall {C : Set} {X : Set -> Set}, AlgF Alg C X -> Alg C X.
Definition unrollAlg : 
  forall {C : Set} {X : Set -> Set}, Alg C X -> AlgF Alg C X.
\end{verbatim}
\caption{The type for algebras (\texttt{Subrec.v})}
\label{fig:algf}
\end{figure}

The type \verb|FoldT Alg C| is the type for fold functions which apply
algebras of type \verb|Alg| to data of type \verb|C|, which we have
already dubbed the \emph{anchor type} of the recursion.  At the top
level of code, the anchor type would just be \verb|List| (for
example).  When one initiates a subsidiary recursion, though, the
anchor type will instead by the abstract type \verb|R| for the outer recursion.

The variable \verb|Alg| occurs only positively (but not strictly
positively) in \verb|AlgF|, because it occurs negatively in
\verb|FoldT Alg R| which occurs negatively in \verb|AlgF Alg C X|.  So
we can indeed take a fixed-point of \verb|AlgF| to define the constant
\verb|Alg|.

Let us look at \verb|AlgF|.  As noted already, each recursion is based
on an abstract type \verb|R|, representing the data upon which we will
recurse.  This is the first argument to a value of type
\verb|AlgF Alg C X|.  An algebra can assume nothing about \verb|R|
except that it supports the following operations.  First there is
\verb|reveal|, which turns an \verb|R| into a \verb|C|.  This reveals
that the data of type \verb|R| are really values of the anchor type
of this recursion.  Next we have \texttt{fold}, which will allow us to fold
another algebra over data of type \verb|R|.  We will use \verb|fold|
to initiate subsidiary recursions.  Then there is \verb|eval|, for
recursive calls on data of type \verb|R|.

As noted already, for subsidiary recursion, algebras have a carrier
\verb|X| which depends (functorially) on a type.  This is so that (i)
inside an inner recursion we may compute a result of some type that
may mention \verb|R|, but (ii) outside that recursion, the result will
mention the anchor type \verb|C|.  The \verb|eval| function returns
something of type \verb|X R|, and so does the algebra itself; this
demonstrates (i).  For (ii): if we look at the definition of
\verb|FoldT| in the figure, we see that folding an algebra of type
\verb|alg C X| over a value of type \verb|C| produces a result of type
\verb|X C|.  Having a functor for the carrier of the algebra gives us
the flexibility to type results inside a recursion with the abstract
type \verb|R|, but view those results as having the anchor type
\verb|C| outside the recursion.

The final definitions in the figure are for \verb|fold|, which allows
us to fold an \verb|Alg| over a value of type \verb|Subrec|; and for mapping
between \verb|Alg| and its unfolding in terms of \verb|AlgF|.  We will
return to the code for \verb|Subrec.v| in Section~\ref{sec:deriv}.

\section{Examples of subsidiary recursion}
\label{sec:examples}

Having seen the interface for subsidiary recursion in Coq, let us
consider now some examples.

\subsection{The \texttt{span} function (\texttt{Span.v})}

Given a predicate \verb|p : A -> bool|, and a value of type
\verb|List A|, we would like to compute a pair of type
\verb|list A * List A|, where the first component is the maximal
prefix whose elements satisfy \verb|p|, and the second is the
remaining suffix.  This is the typing for a top-level recursion.  More
generally, though, given an anchor type \verb|R : Set| along with a
fold function for that anchor type (i.e., of type \verb|FoldT (Alg (ListF A)) R|),
we would like to map an input list of type \verb|R| to a pair of
type \verb|list A * R|.  The first component of this pair is
going to be built up from scratch, and so cannot have type \verb|R|.
But the second component will be a subdatum of the input list,
and so can still have type \verb|R|.  This will enable outer recursions
to continue on that component.  So we want: 
\begin{verbatim}
Definition spanr{R : Set}(fo:FoldT (Alg (ListF A)) R)
                (p : A -> bool)(xs : R) : list A * R.
\end{verbatim}
\noindent From this we can also define the top-level recursion, by
supplying \verb|fold (ListF A)|, which is the function for folding an algebra
over a list (Figure~\ref{fig:algf}), for the argument \verb|fo| of \verb|spanr|:
\begin{verbatim}
Definition span(p : A -> bool)(xs : List A) : list A * List A
  := spanr (fold (ListF A)) p xs.
\end{verbatim} 

Before we define \verb|spanr|, we must resolve a small problem.
If the first element of the input list
\verb|xs| to \verb|span| does not satisfy \verb|p|, then \verb|span|
should return \verb|([], xs)|.  But when recursing on \verb|xs|, we
will see it only in the form of a subdata structure of type
\verb|F R|.  We will not be able to return it from our recursion at
type \verb|R|, and hence we would not be able to return \verb|([],xs)|
as desired.  To work around this, we will have our recursion return a value
of type \verb|SpanF R|:
\begin{verbatim}
Inductive SpanF(X : Set) : Set :=
  SpanNoMatch : SpanF X
| SpanSomeMatch : list A -> X -> SpanF X.
\end{verbatim}
\noindent The idea is that the recursion will signal if it is in the
one tricky case where \verb|p| does not match the first element, by
returning \verb|SpanNoMatch|.  Otherwise, it will be able to return,
via \verb|SpanSomeMatch|, a prefix and the suffix at type
\verb|R|. The prefix will be nonempty, and hence the suffix will be at
most the tail of \verb|xs|.  This tail is available to the algebra
in the subdata structure of type \verb|F R|.

Figure~\ref{fig:span} gives the algebra \verb|SpanAlg| for computing
\verb|span|, and the code for \verb|spanr|.  We elide the proof
\verb|SpanFunctor| that \verb|SpanF| is indeed a \verb|Functor|,
and make \verb|X| implicit in the constructors of \verb|SpanF|.
The type of \verb|SpanAlg p C| is
\begin{verbatim}
Alg (ListF A) C SpanF
\end{verbatim}
This states that we are defining an algebra (\verb|Alg|) for the
\verb|ListF A| functor, with anchor type \verb|C| and carrier
\verb|SpanF|.  \verb|SpanF| has type \verb|Set -> Set|, as required
for the carriers of our algebras. The definition of \verb|SpanAlg| is
actually parametrized by \verb|C|, which is good, as it means we can
use \verb|SpanAlg| for top-level or subsidiary recursions.

Let us continue through the code for \verb|SpanAlg|
(Figure~\ref{fig:span}).  We use \verb|rollAlg| to create an algebra
from something whose type is an application of \verb|AlgF|.  This
takes in all the components of the recursion universe: the abstract
type \verb|R|, the \verb|reveal| function (not needed in this case),
the fold function (\verb|fo|) for any subsidiary recursions (also not
needed here), a function we choose to name \verb|span| for making
recursive calls, and finally \verb|xs : ListF A R|.  The algebra
pattern-matches on this \verb|xs|.  In the cases where it is empty or
where its head (\verb|hd|) does not satisfy \verb|p|, we return
\verb|SpanNoMatch|.  This signals to the caller that we really wished
to return \verb|([],xs)|, but could not because we do not have
\verb|xs| at type \verb|R|.  If the head does satisfy \verb|p|, then
we recurse on the tail (\verb|tl : R|) by calling the provided \verb|span : R -> SpanF R|.
If \verb|span tl|
returns \verb|SpanNoMatch|, that means that we should make \verb|tl|
the suffix in the pair we return (via \verb|SpanSomeMatch|).  Happily,
we have \verb|tl : R| here, so we can do this.  In either case (for
return value of \verb|span tl|), we add the head to the front of the
prefix.  We define \verb|spanhr| to invoke the fold function it is
given, on the algebra (\verb|SpanAlg|).

The final twist is now in the definition of \verb|spanr|.  We call
\verb|spanhr| on the input \verb|xs : R|.  If \verb|spanhr| returns
\verb|SpanNoMatch|, then we are supposed to return \verb|([],xs)|,
which we can do here, because we have \verb|xs : R|.  It was only
inside the algebra that we lost the information that the subdata
structure of type \verb|F R| is derived from a value of type \verb|R|.
If \verb|spanhr| returns \verb|SpanSomeMatch|, then the return value
gives us the nonempty prefix (\verb|l|) and the suffix (\verb|r|), which we then return.

\begin{figure}
\begin{verbatim}
Definition SpanAlg(p : A -> bool)(C : Set)
  : Alg (ListF A) C SpanF :=
  rollAlg (fun R reveal fo span xs => 
     match xs with
         Nil => SpanNoMatch 
       | Cons hd tl =>
          if p hd then
            match (span tl) with
              SpanNoMatch => SpanSomeMatch [hd] tl
            | SpanSomeMatch l r => SpanSomeMatch (hd::l) r
            end
          else
            SpanNoMatch 
       end).

Definition spanhr{R : Set}(fo:FoldT (Alg (ListF A)) R)
                 (p : A -> bool)(xs : R) : SpanF R :=
  fo SpanF SpanFunctor (SpanAlg p R) xs.

Definition spanr{R : Set}(fo:FoldT (Alg (ListF A)) R)
                (p : A -> bool)(xs : R) : list A * R
  := match spanhr fo p xs with
       SpanNoMatch => ([],xs)
     | SpanSomeMatch l r => (l,r)
     end.
\end{verbatim}
\caption{The algebra \texttt{SpanAlg} for the \texttt{span} function, and some functions based on it}
\label{fig:span}
\end{figure}

We can easily define \verb|break|, in Figure~\ref{fig:break}.  The
function \verb|breakr| is a version of \verb|break| that can be used
for subsidiary recursion, similarly to \verb|spanr| for \verb|span|.
Such a function always takes in a fold function (\verb|fo|) with
anchor type \verb|R|, which then is used to fold the algebra in
question.

\begin{figure}
\begin{verbatim}
Definition breakr{R : Set}(fo:FoldT (Alg (ListF A)) R)
                 (p : A -> bool)(xs : R) : list A * R :=
  spanr fo (fun x => negb (p x)) xs.

Definition break(p : A -> bool)(xs : List A) : list A * List A :=
  breakr (fold (ListF A)) p xs.
\end{verbatim}
\caption{The \texttt{break} function and its more flexible version, \texttt{breakr}, defined in terms of \texttt{spanr} (Figure~\ref{fig:span})}
\label{fig:break}
\end{figure}

\subsection{The \texttt{wordsBy} function (\texttt{WordsBy.v})}

Let us now see how to write \verb|wordsBy|, our example function from
Section~\ref{sec:intro}, using \verb|spanr| as a subsidiary recursion.
The code is in Figure~\ref{fig:wordsby}, assuming a type
\verb|A : Set|.  The setup is similar to that for \verb|span|.  We first define
an algebra \verb|WordsBy|, parametrized by anchor type \verb|C| (and also
the predicate \verb|p|), of type
\begin{verbatim}
Alg (ListF A) C (Const (list (list A)))
\end{verbatim}
\noindent This says that \verb|WordsBy p C| is an algebra (\verb|Alg|)
for the \verb|ListF A| functor, with anchor type \verb|C|, and carrier
\verb|Const (list (list A))|.  \verb|Const| is the combinator for
creating the object part of constant functors; \verb|FunConst| creates
the morphism part (i.e., the \verb|fmap| function).  We use it
\verb|Const| here and in other examples where the return type of the
algebra will not depend on its abstract type \verb|R|.  Here, we are
constructing from scratch a list of lists, so it will not be legal to
recurse on the list itself, or its (list) elements.  Intead, we just
use the \verb|list| type of Coq's standard library.

The code for \verb|WordsBy| is, except for the noise of \verb|rollAlg|
and accepting the components of the recursion universe, essentially
the same as what we saw in Section~\ref{sec:intro}.  We pattern match
on \verb|xs : ListF A R|.  Recall that for this function, we are
trying to drop elements which satisfy \verb|p|, and return a list of
the sublists between maximal sequences of such elements.  In the
\verb|Cons| case, if the head (\verb|hd|) satisfies the predicate, then we are
supposed to drop it and recurse.  This is legal, because \verb|tl : R|
and \verb|wordsBy : R -> list (list A)|.  In the \verb|else| case,
we use \verb|breakr| to obtain the maximal prefix \verb|w| of \verb|tl| that
does not satisfy \verb|p|, and the remaining suffix \verb|z|.

Here we see the benefit of our approach.  From Figure~\ref{fig:break},
the return type of \verb|breakr| is \verb|list A * R|, where \verb|R|
is the anchor type of the provided fold function \verb|fo|.  And
\verb|fo| has type \verb|FoldT (ListF A) Alg R|, from the definition
of \verb|AlgF| in Figure~\ref{fig:algf} (instantiating the functor
with \verb|ListF A|).  This means that from the invocation of
\verb|breakr|, we get \verb|w : list A| and \verb|z : R|.  And so we
can indeed apply \verb|wordsBy : R -> list (list A)| to
\verb|z| to recurse.

\begin{figure}
\begin{verbatim}
  Definition WordsBy(p : A -> bool)(C : Set)
    : Alg (ListF A) C (Const (list (list A))) :=
    rollAlg (fun R reveal fo wordsBy xs => 
         match xs with
           Nil => [] 
         | Cons hd tl =>
           if p hd then
             wordsBy tl
           else
             let (w,z) := breakr fo p tl in
             (hd :: w) :: wordsBy z
         end).

Definition wordsByr{R : Set}(fo:FoldT (Alg (ListF A)) R)
                   (p : A -> bool)(xs : R) : list (list A) :=
  fo (Const (list (list A))) (FunConst (list (list A))) (WordsBy p R) xs.

Definition wordsBy(p : A -> bool)(xs : List A) : list (list A) :=
  wordsByr (fold (ListF A)) p xs.
\end{verbatim}
\caption{The \texttt{wordsBy} and \texttt{wordsByr} function, defined using an algebra}
\label{fig:wordsby}
\end{figure}

\subsection{The \texttt{mapThrough} function (\texttt{MapThrough.v})}

The Haskell library \verb|Data.List.Extra| has a function \verb|repeatedly|, which
is defined essentially as follows; I have attempted a more informative name:
\begin{verbatim}
mapThrough :: (a -> [a] -> (b, [a])) -> [a] -> [b]
mapThrough f [] = []
mapThrough f (a:as) = b : mapThrough f as'
    where (b, as') = f a as
\end{verbatim}
\noindent The idea is that the function is like the standard \verb|map| function
on lists, except that here, the function \verb|f| that we are mapping (or ``mapping
through'') takes in not just the current element \verb|a|, but also the tail \verb|as|.
It then returns the value \verb|b| to include in the output list, and whatever other
list it wishes, upon which \verb|mapThrough| will recurse.

We can write this combinatory using our infrastructure for subsidiary recursion.
For this to work, we need to supply the mapped function with the fold function
for \verb|mapThrough|'s recursion.  This is so that the mapped function can
initiate a subsidiary recursion, returning a value in the abstract type \verb|R|
of \verb|mapThrough|'s recursion.  So the type we will use for mapped functions
is:
\begin{verbatim}
Definition mappedT(A B : Set) : Set :=
  forall(R : Set)(fo:FoldT (Alg (ListF A)) R), A -> R -> B * R.
\end{verbatim}
\noindent This type is more informative than the Haskell type,
since it shows that the second component of the returned value
must have type \verb|R|, and hence must be (hereditarily) a tail
of the input of type \verb|R|.

Given this definition, the code for \verb|mapThrough| and
\verb|mapThroughr| is in Figure~\ref{fig:mapthrough}.  The code for
\verb|MapThroughAlg| is very similar (discounting syntax) to the
Haskell code above.  Here, though, when we call \verb|f|, we must
supply the abstract type \verb|R| and fold function \verb|fo|.  Then,
from the definition of \verb|mappedT|, we have that \verb|b : B| and
\verb|c : R|.  So we may indeed invoke \verb|mapThrough : R -> List B|
on \verb|c|.  Note that as we are building up a new list from scratch
(rather than just extracting some tail of the input list), we just
return \verb|list B|; we cannot perform further subsidiary recursion
on the output.  

\begin{figure}
\begin{verbatim}
Definition MapThroughAlg{B : Set}(f:mappedT A B)
           (C : Set) : Alg (ListF A) C (Const (list B)) :=
  rollAlg (fun R reveal fo mapThrough xs => 
    match xs with
      Nil => []
    | Cons hd tl =>
      let (b,c) := f R fo hd tl in
        b :: mapThrough c
    end).

Definition mapThroughr{R : Set}(fo:FoldT (Alg (ListF A)) R)
                      {B : Set}(f:mappedT A B) : R -> list B :=
  fo (Const (list B)) (FunConst (list B)) (MapThroughAlg f R).

Definition mapThrough{B : Set}(f:mappedT A B) : List A -> list B :=
  mapThroughr (fold (ListF A)) f.
\end{verbatim}
\caption{The \texttt{mapThrough} and \texttt{mapThroughr} functions,
  with their defining algebra}
\label{fig:mapthrough}
\end{figure}

\subsection{Run-length encoding (\texttt{Rle.v})}

Using \verb|mapThrough|, we can write a quite concise function for
\emph{run-length encoding}, a basic data-compression algorithm where
maximal sequences of $n$ occurrences of element $e$ are summarized by
the pair $(n,e)$~\cite{datacomp}.  In Haskell, invoking \verb|span| and
\verb|mapThrough| (defined above), the code is simply
\begin{verbatim}
rle :: Eq a => [a] -> [(Int,a)]
rle = mapThrough compressSpan
  where compressSpan a as =
          let (p,s) = span (== a) as in
            ((1 + length p, a),s)
\end{verbatim} 
\noindent (Recall that \verb|(== a)| is a Haskell \emph{section}
testing its input for equality with \verb|a|.)  The
\verb|compressSpan| helper function gathers up all elements at the
start of the tail \verb|as| that are equal to the head \verb|a|.  This
prefix is returned as \verb|p|, with the remaining suffix as \verb|s|.
The pair \verb|(1 + length p, a)| is returned to summarize
\verb|a :: p|.  We then use \verb|mapThrough| to iterate
\verb|compressSpan| through the suffix \verb|s|.

Assuming \verb|A : Set| and an equality test \verb|eqb : A -> A -> bool| on it,
code for run-length encoding using our infrastructure is listed in Figure~\ref{fig:rle}.
The function \verb|compressSpan| is written at the type \verb|mappedT A (nat * A)| that
will be required by \verb|mapThrough|.  Unfolding the definition of \verb|mappedT|,
\verb|compressSpan| has type:
\begin{verbatim}
forall(R : Set)(fo:FoldT (Alg (ListF A)) R), A -> R -> (nat * A) * R.
\end{verbatim}
\noindent It will be invoked by the code for \verb|mapThrough| with a
fold function \verb|fo| with anchor type \verb|R|, and then has the
responsibility of mapping the tail at type \verb|R| (second input) to
a result upon which \verb|mapThrough| should recurse (second component
of the output pair).  Then we define an algebra \verb|RleAlg| by
supplying \verb|compressSpan| as the function to map through, to
\verb|MapThroughAlg| (Figure~\ref{fig:mapthrough}).  Following the
pattern we have seen in all the examples above, we may then define
function \verb|mapThroughr| for subsidiary recursions, and
\verb|mapThrough| for top-level recursions.



\begin{figure}
\begin{verbatim}
Definition compressSpan : mappedT A (nat * A) :=
  fun R fo hd tl => 
    let (p,s) := spanr fo (eqb hd) tl in
       ((succ (length p),hd), s).

Definition RleCarr := Const (list (nat * A)).
Definition RleAlg(C : Set) : Alg (ListF A) C RleCarr :=
  MapThroughAlg compressSpan C.

Definition rle(xs : List A) : list (nat * A)
  := @fold (ListF A) RleCarr (FunConst (list (nat * A))) (RleAlg (List A)) xs.
\end{verbatim}
\caption{The function \texttt{rle} for run-length encoding, and the algebra \texttt{RleAlg} defining it
in terms of \texttt{MapThroughAlg} (Figure~\ref{fig:mapthrough})}
\label{fig:rle}
\end{figure}


\section{Derivation of subsidiary recursion}
\label{sec:deriv}

\subsection{Retractive-positive recursive types}
\label{sec:mu}

As we have seen, our definitions require a form of positive-recursive
types, to allow algebras to accept fold functions that themselves
require algebras, and also for the definition of \verb|Subrec|.  But
as recalled already, full positive-recursive types are incompatible
with Coq's type theory~\cite{coquand88}.  It is worth noting that one
can impose some restrictions on large eliminations which then allow
positive-recursive types~\cite{blanqui05}.  This approach would
require changing the underlying theory.  To avoid this, we here take a
different approach, exploiting Coq's impredicative polymorphism.

This is done in a file \verb|Mu.v|, whose central definitions are in
Figure~\ref{fig:mu}.  The development is parametrized by
\verb|F : Set -> Set| which is assumed to have an \verb|fmap| function
(morphism part of the functor) of type
\begin{verbatim}
forall A B : Set, (A -> B) -> F A -> F B
\end{verbatim}
\noindent which satisfies the identity-preservation law for functors:
\begin{verbatim}
fmapId : forall (A : Set)(d : F A), fmap (fun x => x) d = d
\end{verbatim}

\begin{figure}
\begin{verbatim}
  Inductive Mu : Set := 
    mu : forall (R : Set), (R -> Mu) -> F R -> Mu.

  Definition inMu(d : F Mu) : Mu :=
    mu Mu (fun x => x) d.

  Definition outMu(m : Mu) : F Mu :=
    match m with
    | mu A r d => fmap r d
    end.

  Lemma outIn(d : F Mu) : outMu (inMu d) = d.
\end{verbatim}
\caption{Derivation of retractive-positive recursive types}
\label{fig:mu}
\end{figure}  

Let us consider the code in Figure~\ref{fig:mu}.  The critical
idea is embodied in the definition of \verb|Mu|.  Ideally, we would like
to have a definition like
\begin{verbatim}
  Inductive Mu' : Set := mu' : F Mu' -> Mu'.
\end{verbatim}
\noindent This is exactly what is used in approaches to modular
datatypes in functional programming, like
Swierstra's~\cite{swierstra08}.  But this definition is (rightly)
rejected by Coq, as instantiations of
\verb|F| that are not strictly positive would be unsound.

Instead, the definition of \verb|Mu| in Figure~\ref{fig:mu} weakens
this ideal definition to a strictly positive approximation:
\begin{verbatim}
  Inductive Mu : Set := 
    mu : forall (R : Set), (R -> Mu) -> F R -> Mu.
\end{verbatim}
\noindent Instead of taking in \verb|F Mu|, constructor \verb|mu|
accepts an input of type \verb|F R|, for some type \verb|R| for which
we have a function of type \verb|R -> Mu|.  The impredicative
quantification of \verb|R| is essential here: we instantiate it with
\verb|Mu| itself in the definition of \verb|inMu|
(Figure~\ref{fig:mu}).  So this approach would not work in a
predicative theory like Agda's.  The quantification of \verb|R| can be
seen as applying a technique due to Mendler, of introducing
universally quantified variables for problematic type occurrences, to
a datatype constructor.  We will review this in
Section~\ref{sec:related}.

Returning to Figure~\ref{fig:mu}, we have functions \verb|inMu| and
\verb|outMu|, which make \verb|F Mu| a retraction (\verb|outIn|) of
\verb|Mu|: the composition of \verb|outMu| and \verb|inMu| is
(extensionally) the identity on \verb|F Mu|.  But the reverse
composition cannot be proved to be the identity, because of the basic
problem of \textbf{noncanonicity} that arises with this definition.

For a simple example of noncanonicity, suppose we instantiate \verb|F|
with \verb|ListF| (of Figure~\ref{fig:listf}).  Please note that as
\verb|Mu| is used in our derivation of subsidiary recursion, we will
not instantiate this \verb|F| with the signature functor of a datatype
directly; but this will show the issue in a simple form.  Let us
temporarily define \verb|List A| as \verb|Mu (ListF A)| (again, for
subsidiary recursion we use a different functor than just \verb|ListF|
directly).  The canonical way to define the empty list would be, 
implicitly instantiating \verb|F| to \verb|ListF A|,
\begin{verbatim}
Definition mkNil := mu (List A) (fun x => x) (NilF A)
\end{verbatim}
\noindent But given this, there are infinitely many other equivalent
definitions.  For any \verb|Q : Set|, we could take
\begin{verbatim}
Definition mkNil' := mu Q (fun x => mkNil) (NilF A)
\end{verbatim}
\noindent Since \verb|fmap f (NilF A)| equals just \verb|NilF B| for
\verb|f : A -> B|, if we apply \verb|outMu| (of Figure~\ref{fig:mu})
to \verb|mkNil'| or \verb|mkNil|, we will get \verb|NilF (List A)|.
But critically, \verb|mkNil| and \verb|mkNil'| are not equal, neither
definitionally nor provably.  One can define a function that puts
\verb|Mu| values in normal form by folding \verb|inMu| over them.
Then \verb|mkNil| and \verb|mkNil'| will have the same normal form,
and be equivalent in that sense.  But the fact that they are not
provably equal is what we term noncanonicity.

Noncanonicity leads to some issues, as we turn next to the problem of
inductive reasoning about subsidiary recursions.  With some care,
however, we can avoid pitfalls, leaving us with a form of
positive-recursive type that enables our definitions to go through.

\section{Interface for subsidiary induction}
\label{sec:interfacei}

\section{Examples of subsidiary induction}
\label{sec:examplesi}

\section{Related Work}
\label{sec:related}

\subsection{Termination}
In some tools, like Coq, Agda, and Lean, termination is checked
statically, based on structural decrease at recursive calls.  Others,
like Isabelle/HOL, allow one to write recursions first, and prove
(possibly with automated help) their termination
afterwards~\cite{krauss}.

It has not escaped the notice of designers of ITPs that structural
recursion is not the only form of terminating recursion.  All the
mentioned tools provide support for well-founded recursion, where for
recursive calls, one must show that the parameter of recursion has
decreased in some well-founded order.

Subsidiary recursion can be seen as a generalization of \emph{nested
recursion}, which allows recursive calls of the form \verb|f (f x)|.
In subsidiary recursion, these are generalized to the form
\verb|f (g x)|, where \verb|g| could be \verb|f| or another
recursively defined function.

\subsection{Mendler encoding}

Mendler introduced the basic idea of using universal abstraction to
support compositional termination checking; an accessible source
is~\cite{mendler91}.  This recursor has type
\[
\all{X}{(\all{R}{(R \to X) \to F\ R \to X}) \to \mu\ F\ \to X}
\]
\noindent We have adopted this idea to the constructor of the type
\verb|Mu| (Section~\ref{sec:mu}).  Previous work explored the
categorical perspective on Mendler-style recursion~\cite{uustalu99}.
Others have explored the possibility of using it with negative type
schemes~\cite{ahn11}.

\bibliography{biblio}

\end{document}
